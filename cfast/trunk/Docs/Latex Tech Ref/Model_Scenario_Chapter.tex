\chapter{Model and Scenario Definition}

Sufficient documentation of calculation models is necessary to assess the adequacy of the
scientific and technical basis of the model and the accuracy of the computational procedures for
scenarios of interest. In addition, adequate documentation will help prevent the unintentional
misuse of the model. The documentation in this document follows the guidelines in ASTM
E1355-04 \cite{ASTM:E1355}.

\section{Model Documentation}

\subsection{Name and Version of the Model}

The name of the model is the Consolidated Fire Growth and Smoke Transport Model or CFAST.
The first public release was version 1.0 in June of 1990. This version was restructured from
FAST \cite{Models:FAST} to incorporate the "lessons learned" from the zone model CCFM developed by Cooper
and Forney [\cite{Models:CCFM}, namely that modification is easier and more robust if the components such as the
physical routines are separated from the solver. chapter 4 (Mathematical and Numerical
Robustness) discusses this in more detail. Version 2 was released as a component of Hazard 1.2
in 1994 \cite{Models:HAZARDI}. The first of the 3.x series was released in 1995 and included a vertical flame spread
algorithm, ceiling jets and nonuniform heat loss to the ceiling, spot targets, and heating and
burning of multiple objects (ignition by flux, temperature or time) in addition to multiple
prescribed fires. As it evolved over the next five years, version 3 included smoke and heat
detectors, suppression through heat release reduction, better characterization of flow through
doors and windows, vertical heat conduction through ceiling/floor boundaries, and
non-rectangular compartments. In 2000, version 4 was released and included horizontal heat
conduction through walls, and horizontal smoke flow in corridors. Version 5 improved the
combustion chemistry. Version 6, released in July, 2005, incorporates a more consistent
implementation of vents, fire objects and event processing and includes a graphical user
interface which substantially improves its usability.

The code is written in FORTRAN 90.

\subsection{Type of Model}

CFAST is a model that predicts the environment within compartmented structures resulting from
a fire prescribed by the user. It is an example of the class of models called finite element. This
particular implementation is called a zone model, and essentially the space to be modeled is
broken down to a few elements. The physics of the compartment fire phenomena is driven by
fluid flow, primarily buoyancy. The usual set of elements or zones are the upper and lower gas
layers, partitioning of the wall/ceiling/floor to an element each, one or more plumes and objects such as fires, targets, and detectors. One feature of this implementation of a finite element model
is that the interface between the elements (in this case, the upper and lower gas layers) can move,
with its position defined by the governing equations.

\subsection{Model Developers}

CFAST was developed and is maintained primarily by the Fire Research Division of the
National Institute of Standards and Technology. The developers are Walter Jones, Richard
Peacock, Glenn Forney, Rebecca Portier, Paul Reneke, and John Hoover \footnote{Naval Research Laboratory, Washington, DC 20375.}.

There have been contributions through research and published papers at Worcester Polytechnic
Institute, University of California at Berkeley, VTT of Finland and CITCM of France. An
important guide to development of the model has been from many people around the world who
have provided ideas, suggestions, comments, detailed questions, opinions on what should happen
in particular scenarios, what physics and chemistry are needed and what types of problems must
be addressed by such a model in order to be useful for real world applications.

\subsection{Relevant Publications}

To accompany the model and simplify its use, NIST has developed this Technical Reference Guide \cite{CFAST_Tech_Guide_6}, a User's Guide \cite{CFAST_Users_Guide_6} and a Software and Validation Guide \cite{CFAST_Valid_Guide_6}.  The Technical Reference Guide describes the underlying physical principles and summarizes sensitivity analysis, model validation, and model limitations consistent with ASTM E 1355 \cite{ASTM:E1355}.  The User�s Guide describes how to use the model.  

The U.S. Nuclear Regulatory Commission has published a verification and validation study of five selected fire models commonly used in support of risk-informed and performance-based fire protection at nuclear power plants \cite{NRCNUREG1824_CFAST}. In addition to an extensive study of the CFAST model, the report compares the output of several other models ranging from simple hand calculations to more complex CFD codes such as the Fire Dynamics Simulator (FDS) developed by NIST.

There are documents available (http://cfast.nist.gov) that are applicable to versions 2, 3, 5 as well
as 6 of both the model and user interface.

\subsection{Governing Equations and Assumptions}

For CFAST, as for most zone fire models, the equations solved are for conservation of mass and
energy. The momentum equation is not solved explicitly, except for use of the Bernoulli
equation for the flow velocity at vents. Based on an integration over the volume of an element,
these equations are solved as ordinary differential equations.

There are two assumptions which reduce the computation time dramatically. The first is that
relatively few zones or elements per compartment is sufficient to model the physical situation.
The second assumption is to close the set of equations without using the momentum equation in
the compartment interiors. This simplification eliminates acoustic waves. Though this prevents one from calculating gravity waves in compartments (or between compartments), coupled with
only a few elements per compartment allows for a prediction in a large and complex space very
quickly.

The equations themselves and the algorithms and sub-models used are discussed in detail in
chapter 3.

\subsection{Input Data Required to Run the Model}

All of the data required to run the CFAST model reside in a primary data file, which the user creates.  Some instances may require databases of information on objects, thermophysical properties of boundaries, and sample prescribed fire descriptions.  In general, the data files contain the following information:

\begin{itemize}
\item compartment dimensions (height, width, length)
\item construction materials of the compartment (e.g., concrete, gypsum)
\item material properties (e.g., thermal conductivity, specific heat, density, thickness, heat of combustion)
\item dimensions and positions of horizontal and vertical flow openings such as doors, windows, and vents
\item mechanical ventilation specifications 
\item fire properties (e.g., heat release rate, lower oxygen limit, and species production rates as a function of time) 
\item sprinkler and detector specifications 
\item positions, sizes, and characteristics of targets
\end{itemize}

Sample data files are provided which encompass many of the validation exercises described in
chapter 6 and in the various articles and reports referenced in chapter 6. These examples range
from simple one-compartment simulations to a large multi-story hotel scenario that includes an
elevator shaft and stairwell pressurization. A complete description of the input parameters
required by CFAST can be found in the CFAST User�s Guide \cite{CFAST_Users_Guide_6}. Some of these parameters have default values included in the model, which are intended to be representative for a range of fire scenarios.  

\subsection{Property Data}

Any simulation of a real fire scenario involves prescribing material properties for the walls,
floor, ceiling, and furnishings. CFAST treats all of these materials as homogeneous solids, thus
the physical parameters for many real objects can only be viewed as approximations to the actual
properties. Describing these materials in the input data file is a challenging task for the model
user. Thermal properties for the most common barrier materials used in construction, e.g.
gypsum wall board, are included in a database, thermal.df, included with the model. These
properties come directly from handbook values for typical materials \cite{Incorpera:1981}.

\subsection{Model Results}

The output of CFAST are the sensible variables that are needed for assessing the environment in
a building subjected to a fire. Once the simulation is complete, CFAST produces an output file containing all of the solution variables.  Typical outputs include (but are not limited to) the following:

\begin{itemize}
\item environmental conditions in the room (such as hot gas layer temperature; oxygen and smoke concentration; and ceiling, wall, and floor temperatures)
\item heat transfer-related outputs to walls and targets (such as incident convective, radiated, and total heat fluxes)
\item fire intensity and flame height
\item flow velocities through vents and openings
\item detector and sprinkler activation times
\end{itemize}

There is more extensive discussion of the output in chapter 6 of this technical
reference manual and the user�s guide. The output is always in the metric system of units.

\subsection{Uses and Limitations of the Model}

CFAST has been developed for use in solving practical fire problems in fire protection engineering.  It is intended for use in system modeling of building and building components.  A priori prediction of flame spread or fire growth on objects is not modeled. Rather, the consequences of a specified fire is estimated. It is not intended for detailed study of flow within a compartment, such as is needed for smoke detector siting.  It includes the activation of sprinklers and fire suppression by water droplets.

The most extensive use of the model is in fire and smoke spread in complex buildings.  The efficiency and computational speed are inherent in the few computation cells needed for a zone model implementation.  The use is for design and reconstruction of time-lines for fire and smoke spread in residential, commercial, and industrial fire applications.  Some applications of the model have been for design of smoke control systems.

\begin{itemize}
\item Compartments:  CFAST is generally limited to situations where the compartment volumes are strongly stratified.  However, in order to facilitate the use of the model for preliminary estimates when a more sophisticated calculation is ultimately needed, there are algorithms for corridor flow, smoke detector activation, and detailed heat conduction through solid boundaries.  This model does provide for non-rectangular compartments, although the application is intended to be limited to relatively simple spaces.  There is no intent to include complex geometries where a complex flow field is a driving force.  For these applications, computational fluid dynamics (CFD) models are appropriate.

\item Gas Layers:  There are also limitations inherent in the assumption of stratification of the gas layers.  The zone model concept, by definition, implies a sharp boundary between the upper and lower layers, whereas in reality, the transition is typically over about 10~\% of the height of the compartment and can be larger in weakly stratified flow.  For example, a burning cigarette in a normal room is not within the purview of a zone model.  While it is possible to make predictions within 5~\% of the actual temperatures of the gas layers, this is not the optimum use of the model.  It is more properly used to make estimates of fire spread (not flame spread), smoke detection and contamination, and life safety calculations.

\item Heat Release Rate:  CFAST does not predict fire growth on burning objects. Heat release rate is specified by the user for one or more fire objects. The model does include the ability to limit the specified burning based on available oxygen. There are also limitations inherent in the assumptions used in application of the empirical models.  As a general guideline, the heat release should not exceed about 1 MW/m$^3$.  This is a limitation on the numerical routines attributable to the coupling between gas flow and heat transfer through boundaries (conduction, convection, and radiation).  The inherent two-layer assumption is likely to break down well before this limit is reached.

\item Radiation:  Because the model includes a sophisticated radiation model and ventilation algorithms, it has further use for studying building contamination through the ventilation system, as well as the stack effect and the effect of wind on air circulation in buildings.  Radiation from fires is modeled with a simple point source approximation.  This limits the accuracy of the model near fire sources. Calculation of radiative exchange between compartments is not modeled.

\item Ventilation and Leakage:  In a single compartment, the ratio of the area of vents connecting one compartment to another to the volume of the compartment should not exceed roughly 1/2 m.  This is a limitation on the plug flow assumption for vents.  An important limitation arises from the uncertainty in the scenario specification.  For example, leakage in buildings is significant, and this affects flow calculations especially when wind is present and for tall buildings.  These effects can overwhelm limitations on accuracy of the implementation of the model.  The overall accuracy of the model is closely tied to the specificity, care, and completeness with which the data are provided.

\item Thermal Properties:  The accuracy of the model predictions is limited by how well the user can specify the thermophysical properties.  For example, the fraction of fuel which ends up as soot has an important effect on the radiation absorption of the gas layer and, therefore, the relative convective versus radiative heating of the layers and walls, which in turn affects the buoyancy and flow.  There is a higher level of uncertainty of the predictions if the properties of real materials and real fuels are unknown or difficult to obtain, or the physical processes of combustion, radiation, and heat transfer are more complicated than their mathematical representations in CFAST.
\end{itemize}

User feedback indicates that using CFAST to predict the transport of heat and combustion
products from a prescribed fire is straightforward, easily and quickly accomplished, and the
results are within expectations. Any user of a computer based (numerical) model must be aware
of the assumptions and approximations being employed. Except for those few materials supplied
in the property databases, the user must supply the thermal properties of the materials, and then
assess the performance of the model compared with experiments to ensure that the model is valid
for a specific application. Only then can the model be expected to predict the outcome of fire
scenarios that are similar to those that have actually been tested.

In addition, there are specific limitations and assumptions made in the development of the algorithms.  These are detailed in the discussion of each of these sub-models:

In addition, there are specific limitations and assumptions made in the development of the
algorithms. These are detailed in the discussion of each of these sub-models:

\begin{itemize}
\item section 3.3 on zone model assumptions,
\item section 3.4.1 on prescribed fires,
\item section 3.4.1.3 on the relationship between fires and mass balance,
\item section 3.4.2.1 on the plume entrainment model,
\item section 3.4.3.1 on doorway flows and entrainment at vents,
\item section 3.4.4 on the assumptions made for corridor flow correlations,
\item section 3.4.5.1 on the assumptions made for radiation heat transfer,
\item section 3.6 on the suppression model, and
\item section 3.7.2 on HCl deposition.
\end{itemize}

\section{Scenarios for which the Model is Evaluated in this Document}

CFAST is used for a wide range of buildings of interest, from �glove-box� size compartments, to
complex hotels to the vehicle assembly building at Cape Canaveral. The intended use of ASTM
E1355 \cite{ASTM:E1355} is to validate a specific scenario of interest so that the model can be used for scenarios
similar to the chosen scenario. The intent of this document, however, is to cover a much wider
range of scenarios which encompass the range of acceptable use of the model. Thus, this section
provides a description of this broader range of scenarios as discussed in this technical reference
guide rather than a single, specific scenario of interest for a validation exercise.

\subsection{Description of Scenarios of Interest}

CFAST is designed primarily to predict the environment within compartmented structures which
results from unwanted fires. These can range from very small containment vessels, on the order
of 1 m\superscript{3} to large spaces on the order of 1000 m\superscript{3}. As discussed in the section on limitations and use (2.1.9), the appropriate size fire depends on the size of the compartment being modeled. A
range of such validation exercises is discussed in chapter 6.

\subsection{List of Quantities Predicted by the Model}

CFAST provides a prediction of the gas layer and boundary temperatures, target temperatures,
species concentration (including soot volume fraction), layer height, fire size and flame length,
floor pressure, flow and fire size at vents, and heat flux (both radiative and convective).
There is more extensive discussion of the output in chapter 6 and this technical reference manual
and the user�s guide.

\subsection{Degree of Accuracy Required for Each Output Quantity}

The degree of  accuracy for each output variable  required by the user is  highly  dependent on  the  technical  issues  associated with  the analysis.  The user  must ask: How accurate does  the analysis have to be  to  answer  the  technical  question posed?  Thus,  a  generalized definition of the  accuracy required for each quantity  with no regard as  to the specifics  of a  particular analysis  is not  practical and would be limited in its usefulness.

Returning   to    the   earlier   definitions    of   ``design''   and ``reconstruction,'' fire scenarios, design applications  typically are  more accurate because the heat release rate is prescribed rather than predicted, and the    initial    and    boundary    conditions   are    far    better characterized. Mathematically, a design calculation is an example of a ``well-posed''  problem  in  which   the  solution  of  the  governing equations is  advanced in  time starting from  a known set  of initial conditions and constrained by a known set of boundary conditions.  The accuracy of the results is a function of the fidelity of the numerical solution, which is  largely dependent on the quality of the model inputs. 

A reconstruction is an example of an ``ill-posed'' problem because the outcome  is known  whereas  the initial  and  boundary conditions  are not. There is  no single, unique solution to the  problem. Rather, it is possible to simulate numerous fires that produce the given outcome. There is no right or wrong answer, but rather a small set of plausible fire scenarios that are  consistent with the collected evidence and physical laws incorporated into the model. These simulations are then used to demonstrate why the fire behaved as it did  based on the current understanding of fire physics  incorporated in  the model.  Most  often, the  result of  the
analysis is only  qualitative. If there is any  quantification at all, it could be in the time to reach critical events, like a roof collapse or room flashover.

The CFAST validation guide \cite{CFAST_Valid_Guide_6} includes efforts to date involving well-characterized geometries and prescribed fires. These studies show that  CFAST predictions vary from being within experimental   uncertainty  to  being   about  30~\%   different  than measurements of temperature, heat flux, gas concentration, {\em etc} (see, for example, reference \cite{NRCNUREG1824_CFAST}). In general, this is adequate for its intended uses which are life-safety
calculations and estimation of the environment to which building elements are subjected in a fire
environment. Applied design margins are typically larger than this level of accuracy and may be
appropriate to insure an adequate factor of safety.
