\chapter{Overview}

This chapter provides a general description of the Consolidated Fire and Smoke Transport (CFAST) model following the guidance in ASTM~E1355~\cite{ASTM:E1355}.

\section{Model Documentation}

\subsection{Name and Version of the Model}

The name of the model is the Consolidated Fire Growth and Smoke Transport Model or CFAST. The first public release was version 1.0 in June, 1990. This version was restructured from FAST~\cite{Models:FAST} to incorporate the ``lessons learned'' from the zone model CCFM developed by Cooper and Forney~\cite{Models:CCFM}. Version~2 was released as a component of Hazard~1.2 in 1994~\cite{Models:HAZARDI, Models:HAZARDI_12}. The first of the 3.x series was released in 1995 and included a vertical flame spread algorithm, ceiling jets and non-uniform heat loss to the ceiling, spot targets, and heating and burning of multiple objects (ignition by flux, temperature or time) in addition to multiple prescribed fires. As it evolved over the next five years, version~3 included smoke and heat detectors, suppression through heat release reduction, better characterization of flow through doors and windows, vertical heat conduction through ceiling/floor boundaries, and non-rectangular compartments. In 2000, version~4 was released and included horizontal heat conduction through walls, and horizontal smoke flow in corridors. Version~5 improved the combustion chemistry. Version 6, released in July, 2005, incorporates a more consistent implementation of vents, fire objects and event processing and includes a graphical user interface which substantially improves its usability.

\subsection{Type of Model}

CFAST is a two-zone fire model that predicts the thermal environment within compartmented structures resulting from a fire. Each compartment is divided into an upper and lower gas layer. The fire drives combustion products from the lower to the upper layer via the plume. The temperature within each layer is uniform, and its evolution in time is described by a set of ordinary differential equations.

\subsection{Model Developers}

CFAST was developed and is maintained primarily by the Fire Research Division of the National Institute of Standards and Technology. The developers are Walter Jones, Richard Peacock, Glenn Forney, Rebecca Portier, Paul Reneke, and John Hoover \footnote{Naval Research Laboratory, Washington, DC 20375.}.

There have been contributions through research and published papers at Worcester Polytechnic Institute, University of California at Berkeley, VTT of Finland and CITCM of France. An important guide to development of the model has been from many people around the world who have provided ideas, suggestions, comments, detailed questions, opinions on what should happen in particular scenarios, what physics and chemistry are needed and what types of problems must be addressed by such a model in order to be useful for real world applications.

\subsection{Relevant Publications}

To accompany the model and simplify its use, NIST has developed this Technical Reference Guide \cite{CFAST_Tech_Guide_6}, a User's Guide \cite{CFAST_Users_Guide_6} and a Software and Validation Guide \cite{CFAST_Valid_Guide_6}.  The Technical Reference Guide describes the underlying physical principles and summarizes sensitivity analysis, model validation, and model limitations consistent with ASTM E 1355 \cite{ASTM:E1355}.  The User�s Guide describes how to use the model.

The U.S. Nuclear Regulatory Commission has published a verification and validation study of five selected fire models commonly used in support of risk-informed and performance-based fire protection at nuclear power plants \cite{NRCNUREG1824_CFAST}. In addition to an extensive study of the CFAST model, the report compares the output of several other models ranging from simple hand calculations to more complex CFD codes such as the Fire Dynamics Simulator (FDS) developed by NIST.

There are documents available (http://cfast.nist.gov) that are applicable to versions 2, 3, 5 as well as 6 of both the model and user interface.

\subsection{Governing Equations and Assumptions}

For CFAST, as for most zone fire models, the equations solved are for conservation of mass and energy. The momentum equation is not solved explicitly, except for use of the Bernoulli equation for the flow velocity at vents. Based on an integration over the volume of an element, these equations are solved as ordinary differential equations.

There are two assumptions which reduce the computation time dramatically. The first is that relatively few zones or elements per compartment is sufficient to model the physical situation. The second assumption is to close the set of equations without using the momentum equation in the compartment interiors. This simplification eliminates acoustic waves. Though this prevents one from calculating gravity waves in compartments (or between compartments), coupled with only a few elements per compartment allows for a prediction in a large and complex space very quickly.



\subsection{Input Data Required to Run the Model}

All of the data required to run the CFAST model reside in a primary data file, which the user creates.  Some instances may require databases of information on objects, thermophysical properties of boundaries, and sample prescribed fire descriptions.  In general, the data files contain the following information:
\begin{itemize}
\item compartment dimensions (height, width, length)
\item construction materials of the compartment (e.g., concrete, gypsum)
\item material properties (e.g., thermal conductivity, specific heat, density, thickness, heat of combustion)
\item dimensions and positions of horizontal and vertical flow openings such as doors, windows, and vents
\item mechanical ventilation specifications
\item fire properties (e.g., heat release rate, lower oxygen limit, and species production rates as a function of time)
\item sprinkler and detector specifications
\item positions, sizes, and characteristics of targets
\end{itemize}
The input files are provided for the validation exercises described in the Validation Guide~\cite{CFAST_Valid_Guide_6}. These examples range from simple one-compartment simulations to a large multi-story hotel scenario that includes an elevator shaft and stairwell pressurization. A complete description of the input parameters required by CFAST can be found in the CFAST User's Guide \cite{CFAST_Users_Guide_6}. Some of these parameters have default values included in the model, which are intended to be representative for a range of fire scenarios.

\subsection{Property Data}

Any simulation of a real fire scenario involves prescribing material properties for the walls,
floor, ceiling, and furnishings. CFAST treats all of these materials as homogeneous solids, thus
the physical parameters for many real objects can only be viewed as approximations to the actual
properties. Describing these materials in the input data file is a challenging task for the model
user. Thermal properties for the most common barrier materials used in construction, e.g.
gypsum wall board, are included in a database, thermal.df, included with the model. These
properties come directly from handbook values for typical materials \cite{Incorpera:1981}.

\subsection{Model Results}

The output of CFAST are the sensible variables that are needed for assessing the environment in a building subjected to a fire. Once the simulation is complete, CFAST produces an output file containing all of the solution variables.  Typical outputs include (but are not limited to) the following:
\begin{itemize}
\item environmental conditions in the room (such as hot gas layer temperature; plume centerline temperature; oxygen and smoke concentration; and ceiling, wall, and floor temperatures)
\item heat transfer-related outputs to walls and targets (such as incident convective, radiated, and total heat fluxes)
\item fire intensity and flame height
\item flow velocities through vents and openings
\item detector and sprinkler activation times
\end{itemize}
There is more extensive discussion of the output in chapter 6 of this technical reference manual and the user's guide. The output is always in the metric system of units.

\subsection{Uses and Limitations of the Model} \label{sec:limitssummary}

CFAST has been developed for use in solving practical fire problems in fire protection engineering.  It is intended for use in system modeling of building and building components.  A priori prediction of flame spread or fire growth on objects is not modeled. Rather, the consequences of a specified fire is estimated. It is not intended for detailed study of flow within a compartment, such as is needed for smoke detector siting.  It includes the activation of sprinklers and fire suppression by water droplets.

\begin{itemize}
\item Compartments: CFAST is generally limited to situations where the compartment volumes are strongly stratified.  This model does provide for non-rectangular compartments, although the application is intended to be limited to relatively simple spaces.  There is no intent to include complex geometries where a complex flow field is a driving force.  For these applications, computational fluid dynamics (CFD) models are appropriate.
\item Gas Layers:  The zone model concept, by definition, implies a sharp boundary between the upper and lower layers, whereas in reality, the transition is typically over about 10~\% of the height of the compartment and can be larger in weakly stratified flow.  For example, a burning cigarette in a normal room is not within the purview of a zone model.  
\item Heat Release Rate: CFAST does not predict fire growth on burning objects. Heat release rate is specified by the user for one or more fires. The model does include the ability to limit the specified burning based on available oxygen. As a general guideline, the heat release rate should not exceed about 1~MW/m$^3$ of compartment volume.  This is a limitation on the numerical routines attributable to the coupling between gas flow and heat transfer through boundaries (conduction, convection, and radiation).  The inherent two-layer assumption is likely to break down well before this limit is reached.
\item Radiation from fires is modeled with a simple point source approximation.  This limits the accuracy of the model within a few diameters of the fire. Calculation of radiative exchange between compartments is not modeled.
\item Ventilation and Leakage:  In a single compartment, the ratio of the volume of the compartment to the area of vents connecting the compartment to another should not exceed roughly 2 m.  This is a limitation on the plug flow assumption for vents.  A more important limitation arises from the uncertainty in the scenario specification.  For example, leakage in buildings is significant, and this affects flow calculations especially when wind is present and for tall buildings.  These effects can overwhelm limitations on accuracy of the implementation of the vent flow model.  The overall accuracy of the model is closely tied to the specificity, care, and completeness with which the data are provided.
\item Thermal Properties:  The accuracy of the model predictions is limited by how well the user can specify the thermophysical properties.  For example, the fraction of fuel which ends up as soot has an important effect on the radiation absorption of the gas layer and, therefore, the relative convective versus radiative heating of the layers and walls, which in turn affects the buoyancy and flow.  There is a higher level of uncertainty of the predictions if the properties of real materials and real fuels are unknown or difficult to obtain, or the physical processes of combustion, radiation, and heat transfer are more complicated than their mathematical representations in CFAST.
\end{itemize}


\section{Scenarios for which the Model is Evaluated in this Document}

CFAST is used for a wide range of buildings of interest, from glove-box size compartments, to complex hotels to the vehicle assembly building at Cape Canaveral. The intended use of ASTM~E1355~\cite{ASTM:E1355} is to validate a specific scenario of interest so that the model can be used for scenarios similar to the chosen scenario. The intent of this document, however, is to cover a much wider range of scenarios which encompass the range of acceptable use of the model. Thus, this section provides a description of this broader range of scenarios as discussed in this technical reference guide rather than a single, specific scenario of interest for a validation exercise.

\subsection{Description of Scenarios of Interest}

CFAST is designed primarily to predict the environment within compartmented structures which results from unwanted fires. These can range from very small containment vessels, on the order of 1 m\superscript{3} to large spaces on the order of 1000 m\superscript{3}. As discussed in the section on limitations and use (see section \ref{sec:limitssummary}), the appropriate size fire depends on the size of the compartment being modeled. A range of such validation exercises is discussed in chapter \ref{sec:validationsummary}.

\subsection{List of Quantities Predicted by the Model}

CFAST provides a prediction of the plume centerline, gas layer, and boundary temperatures, target temperatures, species concentration (including soot volume fraction), layer height, fire size and flame length, floor pressure, flow and fire size at vents, and heat flux (both radiative and convective). There is a more extensive discussion of the output in the CFAST user's guide.

\subsection{Degree of Accuracy Required for Each Output Quantity}

The degree of  accuracy for each output variable  required by the user is  highly  dependent on  the  technical  issues  associated with  the analysis.  The user  must ask: How accurate does  the analysis have to be  to  answer  the  technical  question posed?  Thus,  a  generalized definition of the  accuracy required for each quantity  with no regard as  to the specifics  of a  particular analysis  is not  practical and would be limited in its usefulness.

Returning   to    the   earlier   definitions    of   ``design''   and ``reconstruction,'' fire scenarios, design applications  typically are  more accurate because the heat release rate is prescribed rather than predicted, and the    initial    and    boundary    conditions   are    far    better characterized. Mathematically, a design calculation is an example of a ``well-posed''  problem  in  which   the  solution  of  the  governing equations is  advanced in  time starting from  a known set  of initial conditions and constrained by a known set of boundary conditions.  The accuracy of the results is a function of the fidelity of the numerical solution, which is  largely dependent on the quality of the model inputs.

A reconstruction is an example of an ``ill-posed'' problem because the outcome  is known  whereas  the initial  and  boundary conditions  are not. There is  no single, unique solution to the  problem. Rather, it is possible to simulate numerous fires that produce the given outcome. There is no right or wrong answer, but rather a small set of plausible fire scenarios that are  consistent with the collected evidence and physical laws incorporated into the model. These simulations are then used to demonstrate why the fire behaved as it did  based on the current understanding of fire physics  incorporated in  the model.  Most  often, the  result of  the
analysis is only  qualitative. If there is any  quantification at all, it could be in the time to reach critical events, like a roof collapse or room flashover.

The CFAST validation guide \cite{CFAST_Valid_Guide_6} includes efforts to date involving well-characterized geometries and prescribed fires. These studies show that  CFAST predictions vary from being within experimental   uncertainty  to  being   about  30~\%   different  than measurements of temperature, heat flux, gas concentration, {\em etc} (see, for example, reference \cite{NRCNUREG1824_CFAST}). In general, this is adequate for its intended uses which are life-safety calculations and estimation of the environment to which building elements are subjected in a fire environment. Applied design margins are typically larger than this level of accuracy and may be
appropriate to insure an adequate factor of safety.


\section{Review of the Theoretical Development of the Model}

Details of the software quality assurance process for CFAST is included in the Software and Model Evaluation Guide~\cite{CFAST_Valid_Guide_6}. A brief summary is provided here.

CFAST has been subjected to independent review in two ways, internal and external. First, all documents issued by the National Institute of Standards and Technology receive three levels of internal review by members of the staff not involved in the preparation of the report or underlying research. The theoretical basis of CFAST is presented in this document, and is subject to internal review by staff members who are not active participants in the development of the model, but who are members of the Fire Research Division and are considered experts in the fields of fire and combustion. Externally, the theoretical basis for the model has been published in peer reviewed journals~\cite{Jones:1993a, Jones:1985, Jones:1984} and conference proceedings~\cite{Jones:1991}. In addition, CFAST is used worldwide by fire protection engineering firms who review the technical details of the model related to their particular application. Some of these firms also publish in the open literature reports documenting internal efforts to validate the model for a particular use. Many of these studies are discussed in more detail in the present document.

In addition to the formal review, procedures were in place during the development of CFAST to assure the quality of the model.  These procedures included several components:
\begin{itemize}
\item Review of proposed changes to the code by at least two others involved in the development process to insure that a proposed change was consistent with the rest of the CFAST code and was implemented correctly. These reviews, while informal in nature, provided a comprehensive review of the changes to the model during its development.
\item In addition to the normal NIST document review process, the CFAST software was circulated internally to Fire Research Division Staff to allow interested staff members to test the model.
\item For each major release of CFAST, NIST has maintained a history of the source code which goes back to March 1989.  While it is not practical to reconstruct the programs for each release for use with modern software tools and computer operating systems, the source code history allows the developers to examine what changes were made at each release point. This provides detailed documentation of the history of model development and is often useful to understand the impact of changes to sub-models over the development of the model.
\item Once a release of CFAST was approved by NIST, it was announced with a letter to model users which provided a summary of model changes and available documentation.  In essence, these were a condensation of the internal memorandums, without details or printout of specific code changes.  These memorandums provide documentation of the history of the model development.
\end{itemize}
Finally, CFAST has been reviewed and included in industry-standard handbooks such as the SFPE Handbook \cite{Walton:2003} and referenced in specific standards, including NFPA~805 \cite{NFPA805:2004} and NFPA~551 \cite{NFPA551:2004}.

\subsection{Assessment of the Completeness of Documentation}

There are three primary documents on CFAST, this Technical Reference Guide, the User?s Guide \cite{CFAST_Users_Guide_6}, and the Software Development and Model Evaluation Guide \cite{CFAST_Valid_Guide_6}.  This document is the Technical Reference Guide and provides documentation of the governing equations, assumptions, and approximations of the various submodels. It also includes a summary description of the model structure, and numerics.  The Model User?s Guide includes a description of the model input data requirements and model results. The Software Development and Model Evaluation Guide describes the software quality assurance process used in the development and maintenance of the model and includes an extensive discussion of the validation of the model.

The extensive formal review process for all NIST publications in part insures the quality of the CFAST Guides. In addition, the model developers routinely receive feedback from users on the completeness of the documentation and add clarifications when needed. It is estimated that there are several thousand users of CFAST. Before new versions of the model are released, there is a ``beta test'' period in which the users test the new version using the updated documentation. This process is similar, although less formal, to that which most computer software programs undergo. Training courses for use of the model in fire hazard analysis have been developed from the model documentation and presented at training courses worldwide \cite{Barnett:1990}.

\subsection{Assessment of Justification of Approaches and Assumptions}

The technical approach and assumptions of the model have been presented in the peer reviewed scientific literature and at technical conferences. Also, all documents released by NIST are required to go through an internal editorial review and approval process. This process is designed to ensure compliance with the technical requirements, policy, and editorial quality required by NIST. The technical review includes a critical evaluation of the technical content and methodology, statistical treatment of data, uncertainty analysis, use of appropriate reference data and units, and bibliographic references. CFAST manuals are always first reviewed by a member of the Fire Research Division, then by the immediate supervisor of the author of the document, then by the chief of the Fire Research Division, and finally by a reader from outside the division. Both the immediate supervisor and the division chief are technical experts in the field. Once the document has been reviewed, it is then brought before the Editorial Review Board (ERB), a body of representatives from all the NIST laboratories. At least one reader is designated by the Board for each document that it accepts for review. This last reader is selected based on technical competence and impartiality. The reader is usually from outside the division producing the document and is responsible for checking that the document conforms with NIST policy on units, uncertainty and scope. This reader does not need to be a technical expert in fire or combustion.

Besides formal internal and peer review, CFAST is subjected to continuous scrutiny because it is available to the general public and is used internationally by those involved in fire safety design and postfire reconstruction. The source code for CFAST is also released publicly, and has been used at various universities worldwide, both in the classroom as a teaching tool as well as for research. As a result, flaws in the theoretical development and the computer program itself have been identified and fixed. The user base continues to serve as a means to evaluate the model, which is as important to its development as the formal internal and external peer review processes.

\subsection{Assessment of Constants and Default Values}

A comprehensive assessment of the numerical parameters (such as default time step or solution convergence criteria) and physical parameters (such as empirical constants for convective heat transfer or plume entrainment) used in CFAST is not available in one document. Instead, specific parameters have been tested in various verification and validation studies performed at NIST and elsewhere. Numerical parameters are described in this Technical Reference Guide and are subject to the internal review process at NIST, but many physical parameters are extracted from the literature and do not undergo a formal review. In addition, default values for the various model inputs have been specifically reviewed by a professional fire protection engineering university professor to insure appropriate default values and suggested limits for the various input values. The model user is expected to assess the appropriateness of default values provided by CFAST and make changes to the default values if need be.

\subsection{Summary of Model Validation} \label{sec:validationsummary}

CFAST has been subjected to extensive validation studies by NIST and others.  There are two ways of comparing predictive capability with actual events. The first is simply graphing the time series curves of model results with measured values of variables such as temperature. Another approach is to consider the time to critical conditions such as flashover. Making such direct comparisons between theory and experiment provides a sense of whether predictions are reasonable. This chapter provides a review of CFAST validation efforts by NIST and others to better understand the quality of the predictions by the model.

Some of the work has been  performed at NIST, some by its grantees and some by engineering firms using the model.  Because each organization has its  own reasons for  validating the model, the  referenced papers and reports do not follow any particular guidelines. Some of the works only provide  a qualitative assessment  of the model,  concluding that the  model  agreement with  a  particular  experiment  is ``good''  or ``reasonable.'' Sometimes, the conclusion is that the model works well in certain cases, not as well in others. These studies are included in the survey because the references  are useful to other model users who may have a similar application  and are interested in qualitative assessment. It is important to note  that some of the papers point out flaws in early releases of CFAST that have been corrected or improved in more recent  releases. Some of  the issues raised, however,  are still subjects of  active research. Continued updates for CFAST  are greatly influenced  by   the  feedback   provided  by  users,   often  through publication of validation efforts.


A true validation of a model would involve proper statistical treatment of all the inputs and outputs of the model with appropriate experimental data to allow comparisons over the full range of the model. Thus, the comparisons of the differences between model predictions and experimental data discussed here are intentionally simple and vary from test to test and from variable to variable due to the changing nature of the tests and typical use of different variables. Table \ref{tab:Summary_Relative_Diffs} summarizes the Validation comparisons included for the current version of the model detailed in the Software Development and Experimental Evaluation Guide for CFAST \cite{CFAST_Valid_Guide_6}.

\begin{table}

\label{tab:Summary_Relative_Diffs}

\IfFileExists{../Validation_Guide/FIGURES/ScatterPlots/validation_statistics.tex}{\begin{center}
\begin{longtable}{|c|c|c|c|c|c|}
\hline
Quantity & Number of & Number of & $2\widetilde{\sigma}_E$ & $2\widetilde{\sigma}_M$ & Bias \\
         & Datasets  & Points    &                         &                         &      \\ \hline \hline
\endfirsthead
\hline
Quantity & Number of & Number of & $2\widetilde{\sigma}_E$ & $2\widetilde{\sigma}_M$ & Bias \\
         & Datasets  & Points    &                         &                         &      \\ \hline \hline
\endhead
HGL Temperature & 11 & 219 & 0.14 & 0.48 & 1.14 \\ \hline
HGL Temperature: Forced Ventilation & 7 & 91 & 0.14 & 0.39 & 1.25 \\ \hline
HGL Temperature: Natural Ventilation & 8 & 104 & 0.14 & 0.51 & 1.11 \\ \hline
HGL Temperature: No Ventilation & 3 & 22 & 0.14 & 0.69 & 1.32 \\ \hline
HGL Depth & 7 & 53 & 0.13 & 0.45 & 0.98 \\ \hline
Ceiling Jet Temperature & 6 & 208 & 0.10 & 0.44 & 1.23 \\ \hline
Plume Temperature & 4 & 51 & 0.14 & 0.42 & 1.17 \\ \hline
Oxygen Concentration & 6 & 40 & 0.09 & 0.61 & 1.04 \\ \hline
Carbon Dioxide Concentration & 5 & 31 & 0.09 & 0.49 & 0.85 \\ \hline
Smoke Concentration & 1 & 15 & 0.33 & 1.51 & 3.78 \\ \hline
Compartment Over-Pressure & 1 & 9 & 0.40 & 0.84 & 1.30 \\ \hline
Open Compartment Over-Pressure & 2 & 8 & 0.40 & 0.79 & 1.47 \\ \hline
Target Heat Flux & 1 & 100 & 0.20 & 1.30 & 1.01 \\ \hline
Wall Heat Flux & 6 & 121 & 0.20 & 0.75 & 1.21 \\ \hline
Target Temperature & 2 & 73 & 0.14 & 1.37 & 1.44 \\ \hline
Wall Temperature & 5 & 122 & 0.14 & 0.94 & 1.10 \\ \hline
Smoke Alarm Activations & 7 & 125 & 0.33 & 0.98 & 1.05 \\ \hline
Sprinkler Activation Time & 2 & 68 & 0.20 & 0.52 & 0.84 \\ \hline
\end{longtable}
\end{center}
}{\typeout{Error: Missing file FIGURES/ScatterPlots/validation_statistics.tex}}

\end{table}

Four of the quantities were seen to require additional care when using the model to evaluate the given quantity.  This typically indicates limitations in the use of the model.  A few notes on the comparisons are appropriate:

\begin{itemize}
\item CFAST typically predicts plume temperature near to experimental uncertainty, but tends to under-predict temperatures nearer to the fire source and over-predict temperatures farther away.
\item CFAST typically over-predicts smoke concentration.  Predicted concentrations for open-door tests are within experimental uncertainties, but those for closed-door tests are far higher.
\item With exceptions, CFAST predicts cable surface temperatures within experimental uncertainties.  Total heat flux to targets is typically predicted to within about 30~\%, and often under-predicted.  Radiative heat flux to targets is typically over-predicted compared to experimental measurements, with higher relative difference values for closed-door tests.  Care should be taken in predicting localized conditions (such as target temperature and heat flux) because of inherent limitations in all zone fire models.
\item Predictions of compartment surface temperature and heat flux are typically within 10~\% to 30~\%.  Generally, CFAST over-predicts the far-field fluxes and temperatures and under-predicts the near-field measurements.  This is consistent with the single representative layer temperature assumed by zone fire models.
\end{itemize}

CFAST predictions in this validation study were consistent with numerous earlier studies, which show that the use of the model is appropriate in a range of fire scenarios.  The CFAST model has been subjected to extensive evaluation studies by NIST and others.  Although differences between the model and the experiments were evident in these studies, most differences can be explained by limitations of the model as well as of the experiments.  Like all predictive models, the best predictions come with a clear understanding of the limitations of the model and the inputs provided to perform the calculations.



