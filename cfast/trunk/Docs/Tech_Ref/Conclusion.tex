\chapter{Conclusion}

CFAST is a collection of data, computer programs, and documentation which are used to simulate the important time-dependent phenomena describing the character of a compartment fire. The major functions provided include calculation of the buoyancy-driven as well as forced transport of energy and mass through a series of specified compartments and connections (e.g., doors, windows, cracks, ducts), and the resulting temperatures, smoke optical densities, and gas concentrations after accounting for heat transfer to surfaces and dilution by mixing with clean air.

CFAST is a zone model. The basic assumption of all zone fire models is that each compartment can be divided into a small number of control volumes, each of which is internally uniform in temperature and composition. Beyond these basic assumptions, the model typically involves a mixture of established theory (e.g., conservation equations), empirical correlations where there are data but no theory (e.g., flow and entrainment coefficients), and approximations where there are neither (e.g., post-flashover combustion chemistry) or where their effect is considered secondary compared to the �cost� of inclusion (e.g., temperature dependent material properties)..

The predictive equations are based on the fundamental laws of conservation of mass and energy. Empirical correlations are employed to bridge gaps in existing knowledge. Since the necessary approximations required by operational practicality result in the introduction of uncertainties in the results, the user should understand the inherent assumptions and limitations of the programs, and use these programs judiciously - including sensitivity analyses for the ranges of values for key parameters in order to make estimates of these uncertainties.

As discussed in this report, the CFAST model has been subjected to extensive evaluation studies by NIST and others. Although differences between the model and the experiments were evident in these studies, most differences can be explained by limitations of the model as well as of the experiments. Like all predictive models, the best predictions come with a clear understanding of the limitations of the model and of the inputs provided to do the calculations.

CFAST has proven to be fast, robust and reliable. While the focus of the development of the model has been whole building simulations for assessing the effect of fire on a building environ- ment, principally to calculate threats to life safety of occupants and insults to the building structure, it has been used for a wide variety of building and fire scenarios. The simplest use has been to as- certain the sufficiency of an air handling system to extract smoke. The most complex has been an assessment of fire propagation in a high-rise complex. It is also widely used as the fire model in egress calculations and is described as the basis for hazard estimates in the Simulex \cite{Thompson:1996} and Exodus \cite{Gwynne:2000} egress models.

Because of the speed of the model, it is possible to do real parameter studies of the building environment. It is reasonable to do actual parameter studies including the tens of thousands of variations needed for a proper hazard and risk calculation. Even in those cases where more de- tailed predictions are needed (e.g., smoke detector and sprinkler head siting), CFAST provides the capability to scope the problem, in essence doing parameter studies to determine what specific scenario should be addressed by more detailed calculations.