
\chapter{Background}

CFAST is a two-zone fire model that predicts the thermal environment caused by a fire within a compartmented structure. Each compartment is divided into an upper and lower gas layer. The fire drives combustion products from the lower to the upper layer via the plume. The temperature within each layer is uniform, and its evolution in time is described by a set of ordinary differential equations derived from the fundamental laws of mass and energy conservation. The transport of smoke and heat from zone to zone is dictated by empirical correlations. Because the governing equations are relatively simple, CFAST simulations typically require a few tens of seconds of CPU time on typical personal computers.  The formulation of the equations, their solution, and discussion of validation and verification of the code are presented in companion documents \cite{CFAST_Tech_Guide_7, CFAST_Valid_Guide_7}.


\section{Input Data Required to Run the Model}

All of the data required to run the CFAST model reside in a single input file that the user generates. The file consists of the following information:
\begin{itemize}
\item compartment dimensions (height, width, length)
\item lining materials of the floor, walls, and ceiling of each compartment
\item material properties (e.g., thermal conductivity, specific heat, density, thickness, heat of combustion)
\item dimensions and positions of horizontal and vertical flow openings such as doors, windows, and vents
\item mechanical ventilation specifications
\item fire properties (e.g., heat release rate, lower oxygen limit, and species production rates as a function of time)
\item sprinkler and detector specifications
\item positions, sizes, and characteristics of targets
\end{itemize}
The input files are provided for the validation exercises described in the Validation Guide~\cite{CFAST_Valid_Guide_7}. A complete description of the theoretical and empirical formulation of the model can be found in the CFAST Technical Reference Guide~\cite{CFAST_Tech_Guide_7}.

A comprehensive assessment of the numerical parameters (such as default time step or solution convergence criteria) and physical parameters (such as empirical constants for convective heat transfer or plume entrainment) used in CFAST is not available in one document. Instead, specific parameters have been tested in various verification and validation studies performed at NIST and elsewhere. Numerical parameters are described in this Technical Reference Guide and are subject to the internal review process at NIST, but many physical parameters are extracted from the literature and do not undergo a formal review. The model user is expected to assess the appropriateness of default values provided by CFAST and make changes to the default values if need be.


\section{Model Results}

The output of CFAST are the sensible variables that are needed for assessing the environment in a building subjected to a fire. Once the simulation is complete, CFAST produces an output file containing all of the solution variables.  Typical outputs include (but are not limited to) the following:
\begin{itemize}
\item environmental conditions in the room (such as hot gas layer temperature; plume centerline temperature; oxygen and smoke concentration; and ceiling, wall, and floor temperatures)
\item heat transfer-related outputs to walls and targets (such as incident convective, radiated, and total heat fluxes)
\item fire intensity and flame height
\item flow velocities through vents and openings
\item detector and sprinkler activation times
\end{itemize}

Many of the outputs from the CFAST model are relatively insensitive to uncertainty in the inputs for a broad range of scenarios. However, the more precisely the scenario is defined, the more accurate the results will be. Not surprisingly, the heat release rate is the most important variable, because it provides the driving force for fire-driven flows. Other variables related to compartment geometry such as compartment height or vent sizes, while important for the model results, are typically more easily defined for specific design scenarios than fire related inputs.

\section{Model Version}

Each release of CFAST comes with a version number like 7.0.0, where the first number is the major release, the second is the minor release, and the third is the maintenance release. Major releases occur every few years, and as the name implies significantly change the functionality of the model. Minor releases occur more often, and may cause minor changes in functionality. Release notes can help you decide whether the changes should effect the type of applications that you typically do. Maintenance releases are just bug fixes, and should not affect code functionality.

The first public release of CFAST was version 1.0 in June, 1990. This version was restructured from FAST~\cite{Models:FAST} to incorporate the ``lessons learned'' from the zone model CCFM developed by Cooper and Forney~\cite{Models:CCFM}. Version~2 was released as a component of Hazard~1.2 in 1994~\cite{Models:HAZARDI, Models:HAZARDI_12}. The first of the 3.x series was released in 1995 and included a vertical flame spread algorithm, ceiling jets and non-uniform heat loss to the ceiling, spot targets, and heating and burning of multiple objects (ignition by flux, temperature or time) in addition to multiple prescribed fires. As it evolved over the next five years, version~3 included smoke and heat detectors, suppression through heat release reduction, better characterization of flow through doors and windows, vertical heat conduction through ceiling/floor boundaries, and non-rectangular compartments. In 2000, version~4 was released and included horizontal heat conduction through walls, and horizontal smoke flow in corridors. Version~5 improved the combustion chemistry. Version 6, released in July, 2005, incorporates a more consistent implementation of vents, fire objects and event processing and includes a graphical user interface which substantially improves its usability.

The current version of CFAST, version 7, was released in 2015.
